%% start of file `template.tex'.
%% Copyright 2006-2013 Xavier Danaux (xdanaux@gmail.com).
%
% This work may be distributed and/or modified under the
% conditions of the LaTeX Project Public License version 1.3c,
% available at http://www.latex-project.org/lppl/.


\documentclass[11pt,a4paper,sans]{moderncv}        % possible options include font size ('10pt', '11pt' and '12pt'), paper size ('a4paper', 'letterpaper', 'a5paper', 'legalpaper', 'executivepaper' and 'landscape') and font family ('sans' and 'roman')

% moderncv themes
\moderncvstyle{classic}                            % style options are 'casual' (default), 'classic', 'oldstyle' and 'banking'
\moderncvcolor{red}                                % color options 'blue' (default), 'orange', 'green', 'red', 'purple', 'grey' and 'black'
%\renewcommand{\familydefault}{\sfdefault}         % to set the default font; use '\sfdefault' for the default sans serif font, '\rmdefault' for the default roman one, or any tex font name
%\nopagenumbers{}                                  % uncomment to suppress automatic page numbering for CVs longer than one page

% character encoding
\usepackage[utf8]{inputenc}                       % if you are not using xelatex ou lualatex, replace by the encoding you are using
%\usepackage{CJKutf8}                              % if you need to use CJK to typeset your resume in Chinese, Japanese or Korean

% adjust the page margins
\usepackage[scale=0.75]{geometry}
%\setlength{\hintscolumnwidth}{3cm}                % if you want to change the width of the column with the dates
%\setlength{\makecvtitlenamewidth}{10cm}           % for the 'classic' style, if you want to force the width allocated to your name and avoid line breaks. be careful though, the length is normally calculated to avoid any overlap with your personal info; use this at your own typographical risks...

% personal data
\name{Hai}{Tran}
\title{Graduate Software Engineer}
% \address{street and number}{postcode city}{country}% optional, remove / comment the line if not wanted; the "postcode city" and and "country" arguments can be omitted or provided empty
\phone[mobile]{~022~180~0852}                   % optional, remove / comment the line if not wanted
\phone[fixed]{~(04)~976~1228}                    % optional, remove / comment the line if not wanted
% \phone[fax]{+3~(456)~789~012}                      % optional, remove / comment the line if not wanted
\email{hai.quang.tran@gmail.com}                               % optional, remove / comment the line if not wanted
\homepage{www.hai.world}                         % optional, remove / comment the line if not wanted
\extrainfo{www.github.com/haiquangtran}                 % optional, remove / comment the line if not wanted
\photo[94pt][0.4pt]{cropedd}                       % optional, remove / comment the line if not wanted; '64pt' is the height the picture must be resized to, 0.4pt is the thickness of the frame around it (put it to 0pt for no frame) and 'picture' is the name of the picture file
\quote{"The only true wisdom is in knowing you know nothing" \\ -- Socrates}                                 % optional, remove / comment the line if not wanted

% to show numerical labels in the bibliography (default is to show no labels); only useful if you make citations in your resume
%\makeatletter
%\renewcommand*{\bibliographyitemlabel}{\@biblabel{\arabic{enumiv}}}
%\makeatother
%\renewcommand*{\bibliographyitemlabel}{[\arabic{enumiv}]}% CONSIDER REPLACING THE ABOVE BY THIS

% bibliography with mutiple entries
%\usepackage{multibib}
%\newcites{book,misc}{{Books},{Others}}
%----------------------------------------------------------------------------------
%            content
%----------------------------------------------------------------------------------
\begin{document}
%\begin{CJK*}{UTF8}{gbsn}                          % to typeset your resume in Chinese using CJK
%-----       resume       ---------------------------------------------------------
\makecvtitle

\section{About Me}
\cvitem{}{
I recently completed a Bachelor of Engineering (BE) degree, majoring in Software Engineering with First Class Honours. I am a passionate engineer looking to make a positive impact on the world and improve the quality of people’s lives. I recently co-founded a company that provides a solution for the physically restricted and the elderly.} 

\cvitem{}{I enjoy working on both front-end and back-end applications with a strong emphasis on best practices, Object-oriented (OO) design principles, design patterns, user experience, and reusability. Specifically, I want a job focused in mobile development as it is a tool for bringing my ideas into reality. I am dynamic and have an attention to detail which allows me to understand fundamental concepts and learn new skills rapidly.}

\section{Projects}
\cventry{2016 (Present)}{Slick (Switch With A Click)}{}{}{}{
Website: \url{www.goslick.co.nz/}\newline{} 
Facebook: \url{www.facebook.com/solaceSlick}\newline{}
Email: \url{support@goslick.co.nz}\newline{}
Slick provides a solution to the elderly and the physically restricted, allowing them to have more control of their home devices, increasing their independence. Slick flicks a switch on or off through a remote button or mobile application. I have worked on everything software related in this project, including the front-end and back-end of the application. I created the front-end in AngularJS using Yeoman for scaffolding, and created a REST-API for the back-end in Ruby on Rails (RoR) which communicates with a PostgreSQL database. I am currently working on a web-based mobile application and will further expand to provide native solutions in iOs and Android.
}
\cventry{2016 (Present)}{Personal Website}{}{}{}{
Website: \url{www.Hai.World} \newline{}
In my spare time I enjoy reading books. I am currently reading the 7 Habits of Highly Effective People by Stephen Covey and also learning more about design patterns by reading Head First Design Patterns by Eric Freeman et al. In order to internalize the lessons from these books, I am creating a personal website blog that allows me to summarize these books. The website is done in Jekyll which is a static blog generator in RoR using HTML, CSS, Sass and other front-end tools. The website is still in development. 
}
\cventry{2015}{Food Recommender System (Honours Project)}{}{}{}{Overall, the Hybrid CF system was able to outperform every other system in terms of Accuracy, Precision, Area Under the Curve (ROC). In particular, the Hybrid CF approach was able to provide personalised recommendations at an accuracy of 87.06\%, outperforming the popularity baseline which provided non-personalised recommendations at an accuracy of 73.84\%,  fulfilling the goal of“Find Good Items”.}
\cventry{}{More Projects}{}{}{}{To see more projects, please visit my GitHub at \url{www.github.com/haiquangtran.}}


% \section{Key skills}
% \cvitem{\textbf{Learning}}{I have a passion for learning. I am constantly having to learn new content at University, whether it be new process models, frameworks, algorithms or languages. I am currently learning HTML5/CSS to produce a website for my web design course. I am also learning the .NET framework to produce a Windows Phone 8 application for a real client. On top of that, I am learning SQL and databases}
% \cvitem{\textbf{Programming}}{I am passionate about programming as it allows me to bring ideas into reality. In my spare time, I am currently working on a Sticky Notes application in Android, looking to incorporate chat head functionality. I also incorporate my design skills when I program, producing programs that have high cohesion and low coupling which allows for easy implementation and extensibility. I have created programs such as a Tower Defense game, a Maze Solver, a Flappy Birds clone game, a 3D Renderer pipeline and many other programs.}
% \cvitem{\textbf{Design}}{I am experienced in making 2d graphics such as sprites or images in Adobe Illustrator or Adobe Photoshop. I have previously created an interactive data visualization application that allows users to identify trends and filter out netball information easily. This was made using the library d3. In the past I have created images and animations using Processing. I have also created 2d graphics for a series of games that I have created, the most recent being a clone of flappy bird, and a game called YOLO for the event “Pixel Jam”. }
% \cvitem{\textbf{Driven \& ambitious}}{I am a very ambitious and driven person. I am always setting new goals for myself and striving to achieve them. For example, one of my goals for this year was to learn to play the piano/keyboard. During this year I borrowed a keyboard from my friend and learnt how to play a few songs. I also go to the gym on a daily basis in order to relieve stress. I also read daily for at least half an hour to further gain knowledge and learn. }
% \cvitem{\textbf{Perfectionist}}{I always to try my best. I don’t feel comfortable if I do not try my best at everything I do. For example, I don’t feel comfortable submitting an incomplete assignment. I feel that if I let it happen that I will become complacent and that is not what I want for myself.}



% \cvdoubleitem{category 1}{XXX, YYY, ZZZ}{category 4}{XXX, YYY, ZZZ}
% \cvdoubleitem{category 2}{XXX, YYY, ZZZ}{category 5}{XXX, YYY, ZZZ}
% \cvdoubleitem{category 3}{XXX, YYY, ZZZ}{category 6}{XXX, YYY, ZZZ}

\section{Experience}
\subsection{Vocational}
\cventry{2015--Present}{Co-Founder \& Chief Information Officer}{GoSlick}{Wellington}{}{Software Engineer and Scrum Master. This involved organising the team using an agile approach utilizing tools such as Trello, as well as overseeing any software related activities.
\begin{itemize}%
\item Scrum Master:
  \begin{itemize}
      \item Organising the team using Trello and agile methodologies
      \item Daily stand-up meetings
      \item Sprint Retrospectives
      \item Product \& Backlog grooming
  \end{itemize}
\item Software Engineer:
  \begin{itemize}%
  \item Building a responsive website
  \item Creating a RESTful API for the back-end
  \item Web-based mobile application (In Progress)
  \end{itemize}
\item Other:
  \begin{itemize}%
    \item Product development and marketing as well as sending emails and setting up meetings
  \end{itemize} 
\end{itemize}}

\cventry{2014--2015}{Software Engineer Intern}{Solnet Solutions}{Wellington}{}{Description line 1\newline{}Description line 2}

\cventry{2014--2015}{Web Developer}{Energy Management Association of New Zealand}{Wellington}{}{Drupal Developer.}

\subsection{Miscellaneous}

\cventry{2015--2016}{Participant}{Victoria Entrepreneur Bootcamp}{Wellington}{}{Description line 1\newline{}Description line 2}

\cventry{2011--2013}{Casual Worker}{Kelly Services}{Wellington}{}{\begin{itemize}%
\item Resene
\item Imperial Tobacco
  \begin{itemize}%
  \item Test
    \begin{itemize}
    \item Sub-sub-achievement i;
    \end{itemize}
  \item Sub-achievement (c);
  \end{itemize}
\item Achievement 3.
\end{itemize}}

\cventry{2010--2011}{Tutor}{Family Friend}{Wellington}{}{Tutoring Math and English to two children the ages of 11 and 13 for a family friend.}


% \section{Languages}
% \cvitemwithcomment{Language 1}{Skill level}{Comment}
% \cvitemwithcomment{Language 2}{Skill level}{Comment}
% \cvitemwithcomment{Language 3}{Skill level}{Comment}

\section{Education}
\cventry{2012--2015}{Bachelor of Engineering (BE)}{Victoria University}{Wellington}{\textit{First Class Honours}}{ Majoring in Software Engineering at Victoria University. \\ Graduating in May 2016 with First Class Honours. }  % arguments 3 to 6 can be left empty
\cventry{2006--2011}{Naenae College}{Lower Hutt}{Wellington}{\textit{}}{}

\section{Achievements}
\cvlistitem{First Class Honours in Software Engineering (2015)}
\cvlistitem{Dean's List (2015)}
\cvlistitem{Student Representative for Engineering 301 – Project management at Victoria University (2014)}
\cvlistitem{Rotary Club of Eastern Hutt Scholarship (2011)}
\cvlistitem{Whitireia Star Programming Course (2011)}
\cvlistitem{Wellington Institute of Technology Digital Media Certificate (2009)}
\cvlistitem{NCEA Level 1, Level 2, Level 3 Certificate Endorsed with Merit (2009-2011)}
\cvlistitem{Best junior badminton player trophy (2008), Best Senior badminton player trophy (2010), Waiwhetu badminton club trophy for best senior boys' singles player (2011)}
\cvlistitem{Honours Award: Worthy Contribution to School Life (2007-2011)}


% \section{Extra 2}
% \cvlistdoubleitem{Item 1}{Item 4}
% \cvlistdoubleitem{Item 2}{Item 5\cite{book1}}
% \cvlistdoubleitem{Item 3}{Item 6. Like item 3 in the single column list before, this item is particularly long to wrap over several lines.}

\section{References}
\begin{cvcolumns}
  \cvcolumn{Category 1}{\begin{itemize}\item Person 1\item Person 2\item Person 3\end{itemize}}
  \cvcolumn{Category 2}{Amongst others:\begin{itemize}\item Person 1, and\item Person 2\end{itemize}(more upon request)}
  \cvcolumn[0.5]{All the rest \& some more}{\textit{That} person, and \textbf{those} also (all available upon request).}
\end{cvcolumns}

% Publications from a BibTeX file without multibib
%  for numerical labels: \renewcommand{\bibliographyitemlabel}{\@biblabel{\arabic{enumiv}}}% CONSIDER MERGING WITH PREAMBLE PART
%  to redefine the heading string ("Publications"): \renewcommand{\refname}{Articles}
\nocite{*}
\bibliographystyle{plain}
\bibliography{publications}                        % 'publications' is the name of a BibTeX file

% Publications from a BibTeX file using the multibib package
%\section{Publications}
%\nocitebook{book1,book2}
%\bibliographystylebook{plain}
%\bibliographybook{publications}                   % 'publications' is the name of a BibTeX file
%\nocitemisc{misc1,misc2,misc3}
%\bibliographystylemisc{plain}
%\bibliographymisc{publications}                   % 'publications' is the name of a BibTeX file

\clearpage


















% %-----       letter       ---------------------------------------------------------
% % recipient data
% \recipient{Company Recruitment team}{Company, Inc.\\123 somestreet\\some city}
% \date{January 01, 1984}
% \opening{Dear Sir or Madam,}
% \closing{Yours faithfully,}
% \enclosure[Attached]{curriculum vit\ae{}}          % use an optional argument to use a string other than "Enclosure", or redefine \enclname
% \makelettertitle

% Lorem ipsum dolor sit amet, consectetur adipiscing elit. Duis ullamcorper neque sit amet lectus facilisis sed luctus nisl iaculis. Vivamus at neque arcu, sed tempor quam. Curabitur pharetra tincidunt tincidunt. Morbi volutpat feugiat mauris, quis tempor neque vehicula volutpat. Duis tristique justo vel massa fermentum accumsan. Mauris ante elit, feugiat vestibulum tempor eget, eleifend ac ipsum. Donec scelerisque lobortis ipsum eu vestibulum. Pellentesque vel massa at felis accumsan rhoncus.

% Suspendisse commodo, massa eu congue tincidunt, elit mauris pellentesque orci, cursus tempor odio nisl euismod augue. Aliquam adipiscing nibh ut odio sodales et pulvinar tortor laoreet. Mauris a accumsan ligula. Class aptent taciti sociosqu ad litora torquent per conubia nostra, per inceptos himenaeos. Suspendisse vulputate sem vehicula ipsum varius nec tempus dui dapibus. Phasellus et est urna, ut auctor erat. Sed tincidunt odio id odio aliquam mattis. Donec sapien nulla, feugiat eget adipiscing sit amet, lacinia ut dolor. Phasellus tincidunt, leo a fringilla consectetur, felis diam aliquam urna, vitae aliquet lectus orci nec velit. Vivamus dapibus varius blandit.

% Duis sit amet magna ante, at sodales diam. Aenean consectetur porta risus et sagittis. Ut interdum, enim varius pellentesque tincidunt, magna libero sodales tortor, ut fermentum nunc metus a ante. Vivamus odio leo, tincidunt eu luctus ut, sollicitudin sit amet metus. Nunc sed orci lectus. Ut sodales magna sed velit volutpat sit amet pulvinar diam venenatis.

% Albert Einstein discovered that $e=mc^2$ in 1905.

% \[ e=\lim_{n \to \infty} \left(1+\frac{1}{n}\right)^n \]

% \makeletterclosing

% %\clearpage\end{CJK*}                              % if you are typesetting your resume in Chinese using CJK; the \clearpage is required for fancyhdr to work correctly with CJK, though it kills the page numbering by making \lastpage undefined
% %-----       letter       ---------------------------------------------------------


\end{document}

%% end of file `template.tex'.
